\documentclass{article}
\usepackage[utf8]{inputenc}
\usepackage{amsmath} % assumes amsmath package installed
\usepackage{amssymb}  % assumes amsmath package installed


\begin{document}

%\maketitle

%\section{Introduction}
\section{Basic Conceptions of Petri Nets}
A Petri net structure is a four-tuple $N=(P,T,Pre,Post)$, where $P$ is a finite set of places, graphically represented by circles; $T$ is a finite set of transitions, graphically represented by bars; the $pre$- incidence matrices of $N$ is defined by $Pre: P \times T \to \mathbb{N}$, and the $post$- incidence matrices of $N$ is defined by $Post: P\times T \to\mathbb{N}$, which specify the structure of the net, where $ \mathbb{N} $ is the set of non-negative integers. Specifically, for $p\in P$, $t\in T$ and $k\in \mathbb{N}$, $Pre(p,t)=k$ means that there exists an arc with weight $k$ from $ p $ to $ t $, 0 otherwise; $Post(p,t)=k$ means that there exists an arc with weight $k$ from $ t $ to $ p $, 0 otherwise. The incidence matrix of a Petri net is defined as $C=Post-Pre$.
	
Given two nodes of a Petri net $x\in T$, and $y\in P$, a pre-set of places of $x$ is defined by $^\bullet x=\{y \mid Pre(y,x)>0\}$, and a pro-set of places of $x$ is defined by $x^\bullet=\{y \mid Post(x,y)>0\}$. On the other hand, the definitions of $^\bullet y$ and $y^\bullet $ are analogous. If there are no oriented cycles in this net system, then this Petri net is said to be {\it acyclic}.
	
A marking is a mapping $M\colon P\to\mathbb{N}$, which can be represented by a vector due to the finite place set. An entry in a (marking) vector indicates the number of tokens in a place. We denote by $M(p)$ the marking of place $p$ at marking $M$. A marking $M$ is also denoted as $M=\sum_{p\in P}M(p).p$. A Petri net system is represented as $\left\langle N, M_0\right\rangle$, where $N$ and $M_0$ are a net structure and an initial marking, respectively.
	
At a marking $M$, a transition $t$ is said to be enabled if $M\geq Pre(\cdot,t)$. The firing of an enabled transition $t$ at marking $M$ yields a marking $M’= M+C(\cdot,t)$. We denote that a transition sequence of transitions $\sigma$ is enabled at $M$ by $M[\sigma\rangle$, where $\sigma\in T^*$. The firing of $\sigma$ yields $M’$ from $M$ is denoted by $M[\sigma\rangle M’$. The language $L(N, M_0)$ of a net system is defined as
	 \begin{center}
	 	$L(N, M_0)=\{\sigma\in T^*\mid M_0[\sigma\rangle\} $
	 \end{center}
which is a set of the sequences that are feasible in the net system $\left \langle N, M_0 \right\rangle$. We use $t\in\sigma$ to represent a transition sequence $\sigma$ that includes a transition $t$, which is written as $t\in\sigma$ by a slight abuse of notation. %Given $\sigma\in T^*$, the set $\{t\in T\mid t\in\sigma\}$ is called the support of $\sigma$.
 
A function is defined by $\pi:T^*\to \mathbb{N}^n$, and that associates a transition sequence $\sigma\in T^*$ with a vector $y=\pi(\sigma)\in\mathbb{N}^n$, called its Parikh vector. In other words, for $y=\pi(\sigma)$, $y(t)=k$ means that a transition $t$ appears $k$ times in $\sigma$.

We say a marking $M$ is reachable in a net system if there is a transition sequence $\sigma$ such that $M_0[\sigma\rangle M$. We use $R(N,M_0)$, called the reachability set, to denote the set of all markings reachable from $M_0$ in a net system $\left \langle N, M_0 \right \rangle$, i.e., $R(N,M_0)=\{M\in\mathbb{N}^{|P|}\mid \exists\sigma\in T^*: M_0[\sigma\rangle M\}$. The behavior of $\left\langle N, M_0 \right\rangle$ can be represented by its reachability graph $G(N, M_0)$ that is a digraph starting from the initial marking $M_0$, where a node is a markings in $R(N,M_0)$ and an edge taking $(M,M^\prime)$ labeled with $t$ if $M[t\rangle M^\prime$ holds.

We say a net system $\left \langle N, M_0 \right \rangle$ is a bounded net if there exists a integer $K$, and $K\in \mathbb{N}^+$, such that, for all reachable markings $M\in R(N,M_0)$, and for all $p\in P$, $M(p)\leq K$. The Petri net system is unbounded if this is not the case, i.e. if the number of tokens in at least one place can expand indefinitely. For an unbounded Petri net, it can be analyzed by its coverability set $CS(N, M_0)$ and coverability graph $CG(N, M_0)$.
\end{document}
